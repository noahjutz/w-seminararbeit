% (c) Noah Jutz
% All rights reserved.

\documentclass{article}
\usepackage[utf8]{inputenc}
\usepackage[german]{babel}
\usepackage{graphicx}
\usepackage{float}
\usepackage{wrapfig,lipsum,booktabs}
\usepackage{subfiles}
\usepackage{biblatex,csquotes}
\addbibresource{main.bib}

\graphicspath{ {./images/} }

\title{Entwicklung eines IR-Spektrometers}
\author{Noah Jutz}
\date{}

\begin{document}

\maketitle
\tableofcontents

\newpage
\section{Einführung}
\subfile{sections/einfuehrung.tex}

\newpage
\section{Theoretische Grundlagen}

% 2. Übergang
Um die Funktionsweise eines IR-Spektrometers zu verstehen, müssen zunächst folgende Theoretische Grundlagen aufgeklärt werden.

\subsection{Interferenz}
\subfile{sections/interferenz.tex}

\newpage
\subsection{Doppelspaltexperiment}
\subfile{sections/doppelspalt.tex}

\newpage
\subsection{Absorption}

% Warum Grundlage?

\section{IR-Spektrometer}

% Übergang

\subsection{Komponenten}

\subsubsection{Strahlquelle}

% λ × T = b
% Lötkolben

\subsubsection{Abbildende Optik}

% Spiegel

\subsubsection{Gitter}

% 600 Linien / mm
% Leypold
% Reflexionsgitter

\section{Versuch und Ergebnisse}

% b = k × λ ÷ sin(α)
%   α = 30°; λ = 530nm
%   -> b = 1500nm

\subsection{Versuchsaufbau}

% Bilder:
%   - loetkolben
%   - hg-lampe
%   - spiegel
%   - sensor
%   - gitter

\subsection{Ergebnisse}

% Diagramme
% Fehler

\section{Arduino} % Später

\subsection{Motorsteuerung}

\subsection{Signalerfassung}

\newpage
\printbibliography[heading=bibintoc]

\end{document}
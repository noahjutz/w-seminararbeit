\documentclass{article}
\usepackage[utf8]{inputenc}
\usepackage[german]{babel}

\title{Entwicklung eines IR-Spektrometers}
\author{Noah Jutz}
\date{}

\begin{document}

\maketitle
\tableofcontents


\section{Einführung}

% Fraunhoferlinien in Verbindung mit Infrarot-Spektroskopie
% - Was sind die linien?
% - Entdeckung
% - Zweck
% - Zusammenhang mit IR-Spektroskopie
%   - |Fraunhoferlinien                   |IR-Spektroskopie
%     |-----------------------------------|-----------------------------
%     |Sichtbares licht                   |Unsichtbares Infrarotlicht
%     |chem. Zusammensetzung von Sternen  |Nachweis von Moleküstrukturen
%     |absorption von licht               |absorption vor IR-Strahlung
%     |zerlegung von Sonnenlicht          |zerlegung von IR-Strahlung
%   - Beides Spektroskope

\section{Theoretische Grundlagen}

% Übergang

\subsection{Interferenz}

% Warum Grundlage?

\subsection{Doppelspaltexperiment}

% Warum Grundlage?
% Verbindung zu Interferenz

\subsection{Absorption}

% Warum Grundlage?

\section{IR-Spektrometer}

% Übergang

\subsection{Komponenten}

\subsubsection{Strahlquelle}

% λ × T = b
% Lötkolben

\subsubsection{Abbildende Optik}

% Spiegel

\subsubsection{Gitter}

% 600 Linien / mm
% Leypold
% Reflexionsgitter

\section{Versuch und Ergebnisse}

% b = k × λ ÷ sin(α)
%   α = 30°; λ = 530nm
%   -> b = 1500nm

\subsection{Versuchsaufbau}

% Bilder:
%   - loetkolben
%   - hg-lampe
%   - spiegel
%   - sensor
%   - gitter

\subsection{Ergebnisse}

% Diagramme
% Fehler

\section{Arduino} % Später

\subsection{Motorsteuerung}

\subsection{Signalerfassung}

\end{document}
% (c) Noah Jutz
% All rights reserved.

\documentclass{article}
\usepackage[utf8]{inputenc}
\usepackage[german]{babel}
\usepackage{graphicx}
\usepackage{float}
\usepackage{wrapfig,lipsum,booktabs}
\usepackage{subfiles}
\usepackage{biblatex,csquotes}
\usepackage{fancyhdr}
\usepackage{tabularx}
\usepackage{geometry}
\usepackage{siunitx}

\addbibresource{main.bib}
\addbibresource{images.bib}

\graphicspath{ {./images/} }

\title{Entwicklung eines IR-Spektrometers}
\author{Noah Jutz}
\date{}

% Page numbering
\renewcommand{\headrulewidth}{0pt}
\pagestyle{fancy}
\fancyhf{}
\chead{\thepage}

% Line height
\renewcommand{\baselinestretch}{1.5}

% Page margins
\geometry{
    a4paper,
    lmargin=4cm,
    rmargin=2cm,
    tmargin=4cm,
    bmargin=4cm
}

\begin{document}

\maketitle
\thispagestyle{empty}

\newpage
\tableofcontents
\thispagestyle{empty}

\newpage
\pagenumbering{arabic}
\section{Einführung}
\subfile{sections/1_einfuehrung.tex}

\newpage
\section{Theoretische Grundlagen}
\subfile{sections/2_grundlagen.tex}

\newpage
\section{IR-Spektrometer}
\subfile{sections/3_spektrometer.tex}

\section{Versuch und Ergebnisse}
\subfile{sections/4_versuch.tex}

\section{Arduino}
\subfile{sections/5_arduino.tex}

\newpage
\printbibliography[heading=bibintoc, keyword={literatur}]
\printbibliography[heading=subbibintoc, keyword={image}, title={Bilder}]

\newpage
\subfile{sections/unterschrift.tex}

\end{document}
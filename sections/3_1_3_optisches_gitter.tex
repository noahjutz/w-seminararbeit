\documentclass[../main.tex]{subfiles}

Das Gitter beugt die Strahlung in die verschiedenen Wellenlängen. Hierfür gibt es Gitter verschiedener Linienzahlen pro mm, oder auch Reflexionsgitter, welche sowohl beugen als auch reflektieren (Abb. \ref{fig:gitter_600l}). Um das geeignete Gitter zu finden, werden 2 verschiedene Gitter von jeweils 
$600\frac{\text{Linien}}{\text{mm}}$ und $100\frac{\text{Linien}}{\text{mm}}$
 mit einem Grünen Laser ($\lambda = \SI{530}{\nano\metre}$) getroffen.
\begin{figure}[ht]
    \centering
    \begin{subfigure}[b]{0.45\textwidth}
        \begin{tikzpicture}
            \coordinate (bslit) at (3,3);
            \coordinate (tslit) at (3,5);
            \coordinate (max) at (6,6);
            \coordinate (intersect) at (4,4);

            % Parallel Waves
            \foreach \i in {1,1.5,...,2.5}
                \draw[gray,dashed,thick] ({\i},2) -- ({\i},6);

            % Double Slit @ 3 & 5
            \draw[thick] (3,1.5) -- (3,2.9);
            \draw[thick] (3,3.1) -- (3,4.9);
            \draw[thick] (3,5.1) -- (3,6.5);
            
            % Sine waves
            \draw[gray,very thin,wavy] (tslit) -- (max);
            \draw[blue,thick] (tslit) -- (max);
            
            \draw[gray,very thin,wavy] (bslit) -- (max);
            \draw[blue,thick] (bslit) -- (max);
            
            % Added guidelines
            \draw[gray,dashed,thick] (tslit) -- (intersect);
            %\draw[gray,dashed,thick] (tslit) -- (6,5);
            
            % Labeling
            \draw (3,4.7) node[anchor=north west] {$\alpha$};
            \draw (3.5,3.5) node[anchor=north west] {$k \lambda$};
            \draw (6.2,6) node[anchor=west] {$k$};

            \draw[thick,<->] (2.5,3) -- (2.5,5);
            \draw (2.5,4) node[anchor=east] {$b$};

            % Interference pattern
            \foreach \i in {1,2,...,20}
                \shade[black,white,shading angle={mod(\i,20)*180}] (6,{\i/4+1.25}) rectangle (6.2,{\i/4+1.5});
        \end{tikzpicture}
        \caption{Am Doppelspalt}
        \label{fig:differenz_doppelspalt}
    \end{subfigure}
    \begin{subfigure}[b]{0.45\textwidth}
        \begin{tikzpicture}
            \coordinate (max) at (6,6);
            \coordinate (gittermitte) at (3,4);

            % Interference pattern
            \foreach \i in {1,2,...,20}
                \shade[black,white,shading angle={mod(\i,20)*180}] (6,{\i/4+1.25}) rectangle (6.2,{\i/4+1.5});

            \draw (1,3.75) rectangle (2,4.25);
            \draw (1.5,3.75) node[anchor=north] {Laser};

            \draw[very thick,dashed] (3,3) -- (3,5);
            \draw (3,3) node[anchor=north] {Gitter};
        
            % Sine wave
            \draw[gray,very thin,wavy] (gittermitte) -- (max);
            \draw[blue,ultra thick] (gittermitte) -- (max);
            \draw[gray,very thin,wavy] (2,4) -- (3,4);
            \draw[blue,ultra thick] (2,4) -- (3,4);

            % Labels
            \draw[gray,thick,dashed] (gittermitte) -- (6,4) -- (max) -- (gittermitte);
            \draw (4.5,4) node[anchor=south] {$l$};
            \draw (6,5) node[anchor=east] {$h$};
            \draw (3.5,4) node[anchor=south west] {$\alpha$};
            \draw (6.2,6) node[anchor=west] {$k$};
        \end{tikzpicture}
        \caption{Am Gitter}
        \label{fig:differenz_gitter}
    \end{subfigure}
    \caption{Entstehung von Differenz}
\end{figure}

\noindent Folgende Formel beschreibt die position der Maxima (vielfache von $\lambda$) auf Basis der Strichbreite $b$:
\begin{equation}
    k \lambda = b \sin{\alpha}
\end{equation}
Schreibt man diese um, so erhält man die Linienbreite $b$ des Gitters:
\begin{equation}
    b = \frac{k \lambda}{\sin{\alpha}}
\end{equation}
Der Winkel $\alpha$ lässt sich mit dem Arkustangens des Abstandes zur Wand $l$ und zum Maxima $h$ berechnen (Abb. \ref{fig:differenz_gitter}):
\begin{equation}
    \begin{split}
        b={}&\frac{k \lambda}{\sin{\alpha}} \\
        &= \frac{k \lambda}{\sin(\arctan(\tan(\alpha)))} \\
        &= \frac{k \lambda}{\sin(\arctan(\frac{l}{h}))} \\
    \end{split}
\end{equation}
Bei einem Abstand zur Wand von $l = 100cm$ und einem Laser mit einer Wellenlänge von $\lambda = 530nm$ ist der Abstand vom 0. zum 1. Maximum bei den Gittern von 600 und 100 Linien pro mm jeweils $h_{600} = 34cm$ und $h_{100} = 5cm$. Setzt man diese Werte in die obige Formel ein, erhält man $b_{600}$ und $b_{100}$:

\begin{subequations}
    \noindent\begin{minipage}{0.5\linewidth}
      \begin{align}
          \begin{split}
                b_{600}={}&\frac{k \lambda}{\sin(\arctan(\frac{l}{h}))} \\
                &= \frac{530nm}{\sin{\arctan{\frac{34cm}{100cm}}}} \\
                &\approx \SI{1.6}{\micro\metre}
          \end{split}
      \end{align}
    \end{minipage}%
    \begin{minipage}{0.5\linewidth}
      \begin{align}
          \begin{split}
                b_{100}={}&\frac{k \lambda}{\sin(\arctan(\frac{l}{h}))} \\
                &= \frac{530nm}{\sin{\arctan{\frac{5cm}{100cm}}}} \\
                &\approx \SI{11}{\micro\metre}
          \end{split}
      \end{align}
    \end{minipage}
\end{subequations}

\SI{11}{\micro\metre} eignen sich im Infrarotbereich sehr gut. Daher arbeiten wir mir dem Gitter mit $100 \frac{\text{Linien}}{\text{mm}}$.

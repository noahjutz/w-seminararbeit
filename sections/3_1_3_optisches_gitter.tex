\documentclass[../main.tex]{subfiles}

Das Gitter beugt die Strahlung in die verschiedenen Wellenlängen. Hierfür gibt es Gitter verschiedener Linienzahlen pro mm, oder auch Reflexionsgitter, welche sowohl beugen als auch reflektieren (Abb. \ref{fig:gitter_600l}).
Folgende Formel beschreibt die position der Maxima (vielfache von $\lambda$) auf Basis der Strichbreite $b$. 
\[
    k \lambda = b \sin{\alpha}
\]

\begin{figure}[ht]
    \centering
    \begin{tikzpicture}
        % Interference pattern
        \foreach \i in {1,2,...,20}
            \shade[black,white,shading angle={mod(\i,20)*180}] (6,{\i/4+1.25}) rectangle (6.2,{\i/4+1.5});

        \draw (1,3.75) rectangle (2,4.25);
        \draw (1.5,3.75) node[anchor=north] {Laser};

        \draw[gray,very thin,wavy] (2,4) -- (3,4);
        \draw[blue,ultra thick] (2,4) -- (3,4);
      
        \draw[very thick,dashed] (3,3) -- (3,5);
        \draw (3,3) node[anchor=north] {Gitter};
        
        \coordinate (max) at (6,6);
        \coordinate (gittermitte) at (3,4);
        
        % Sine wave
        \draw[gray,very thin,wavy] (gittermitte) -- (max);
        \draw[blue,ultra thick] (gittermitte) -- (max);

        % Additional guidelines, labeling
        \draw[gray,thick,dashed] (gittermitte) -- (6,4) -- (max) -- (gittermitte);

        \draw (4.5,4) node[anchor=south] {$l$};
        \draw (6,5) node[anchor=east] {$h$};
        \draw (3.5,4) node[anchor=south west] {$\alpha$};
        \draw (6.2,6) node[anchor=west] {$k$};
    \end{tikzpicture}
    \caption{Grafische Darstellung der Formel}
    \label{fig:doppelspalt_beugung_berechnen}
\end{figure}

\begin{subequations}
    \noindent\begin{minipage}{0.5\linewidth}
      \begin{align}
          \begin{split}
              b_{600}={}&\frac{k \lambda}{\sin{\alpha}} \\
              &= \frac{530nm}{\sin{\arctan{\tan{\alpha}}}} \\
              &= \frac{530nm}{\sin{\arctan{\frac{34cm}{100cm}}}} \\
              &\approx \frac{530nm}{\sin{\SI{0.328}{\degree}}} \\
              &\approx \SI{1.6}{\micro\metre}
          \end{split}
      \end{align}
    \end{minipage}%
    \begin{minipage}{0.5\linewidth}
      \begin{align}
          \begin{split}
              b_{100}={}&\frac{k \lambda}{\sin{\alpha}} \\
              &= \frac{530nm}{\sin{\arctan{\tan{\alpha}}}} \\
              &= \frac{530nm}{\sin{\arctan{\frac{5cm}{100cm}}}} \\
              &\approx \frac{530nm}{\sin{\SI{0.05}{\degree}}} \\
              &\approx \SI{11}{\micro\metre}
          \end{split}
      \end{align}
    \end{minipage}
\end{subequations}


% 600 Linien / mm
% Leybold
% Reflexionsgitter

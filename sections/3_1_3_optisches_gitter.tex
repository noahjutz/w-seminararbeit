\documentclass[../main.tex]{subfiles}

Das Gitter beugt die Strahlung in die verschiedenen Wellenlängen. Hierfür gibt es Gitter verschiedener Linienzahlen pro mm, oder auch Reflexionsgitter, welche sowohl beugen als auch reflektieren (Abb. \ref{fig:gitter_600l}).
Folgende Formel beschreibt die position der Maxima (vielfache von $\lambda$) auf Basis der Strichbreite $b$. 
\[
    k \lambda = b \sin{\alpha}
\]

\begin{figure}[ht]
    \centering
    \begin{subfigure}[b]{0.45\textwidth}
        \begin{tikzpicture}
             Parallel Waves
            \foreach \i in {1,1.5,...,2.5}
                \draw[gray,dashed,thick] ({\i},2) -- ({\i},6);

            % Double Slit @ 3 & 5

            \draw[thick] (3,1.5) -- (3,2.9);
            \draw[thick] (3,3.1) -- (3,4.9);
            \draw[thick] (3,5.1) -- (3,6.5);
            
            
            \coordinate (bslit) at (3,3);
            \coordinate (tslit) at (3,5);
            \coordinate (max) at (6,6);
            \coordinate (intersect) at (4,4);
            
            % Sine waves
            \draw[gray,very thin,wavy] (tslit) -- (max);
            \draw[blue,thick] (tslit) -- (max);
            
            \draw[gray,very thin,wavy] (bslit) -- (max);
            \draw[blue,thick] (bslit) -- (max);
            
            % Added guidelines
            \draw[gray,dashed,thick] (tslit) -- (intersect);
            \draw[gray,dashed,thick] (tslit) -- (6,5);
            
            % Labeling
            \draw (3,4.7) node[anchor=north west] {$\alpha$};
            \draw (3.5,3.5) node[anchor=north west] {$k \lambda$};
            \draw (6.2,6) node[anchor=west] {$k$};

            \draw[thick,<->] (2.5,3) -- (2.5,5);
            \draw (2.5,4) node[anchor=south east] {$b$};

            \draw[thick,<->] (3,2) -- (6,2);
            \draw (4.5,2) node[anchor=north] {$l$};

            \draw[thick,<->] (6.75,4) -- (6.75,6);
            \draw (6.75,5) node[anchor=west] {$h$};
            
            % Interference pattern
            \foreach \i in {1,2,...,20}
                \shade[black,white,shading angle={mod(\i,20)*180}] (6,{\i/4+1.25}) rectangle (6.2,{\i/4+1.5});
        \end{tikzpicture}
        \caption{Visualisierung der Formel\newline}
        \label{fig:doppelspalt_beugung_berechnen}
    \end{subfigure}
    \begin{subfigure}[b]{0.45\textwidth}
        \begin{tikzpicture}
            % Interference pattern
            \foreach \i in {1,2,...,20}
                \shade[black,white,shading angle={mod(\i,20)*180}] (6,{\i/4+1.25}) rectangle (6.2,{\i/4+1.5});

            \draw (1,3.75) rectangle (2,4.25);
            \draw (1.5,3.75) node[anchor=north] {Laser};

            \draw[gray,very thin,wavy] (2,4) -- (3,4);
            \draw[blue,ultra thick] (2,4) -- (3,4);
        
            \draw[very thick,dashed] (3,3) -- (3,5);
            \draw (3,3) node[anchor=north] {Gitter};
            
            \coordinate (max) at (6,6);
            \coordinate (gittermitte) at (3,4);
            
            % Sine wave
            \draw[gray,very thin,wavy] (gittermitte) -- (max);
            \draw[blue,ultra thick] (gittermitte) -- (max);

            % Additional guidelines, labeling
            \draw[gray,thick,dashed] (gittermitte) -- (6,4) -- (max) -- (gittermitte);

            \draw (4.5,4) node[anchor=south] {$l$};
            \draw (6,5) node[anchor=east] {$h$};
            \draw (3.5,4) node[anchor=south west] {$\alpha$};
            \draw (6.2,6) node[anchor=west] {$k$};
        \end{tikzpicture}
        \caption{Schematischer Aufbau des Versuchs zur Ermittlung von $b$}
        \label{fig:doppelspalt_beugung_berechnen}
    \end{subfigure}
\end{figure}

\begin{equation}
    \begin{split}
        b={}&\frac{k \lambda}{\sin{\alpha}} \\
        &= \frac{k \lambda}{\sin(\arctan(\tan(\alpha)))} \\
        &= \frac{k \lambda}{\sin(\arctan(\frac{l}{h}))} \\
    \end{split}
\end{equation}

\begin{subequations}
    \noindent\begin{minipage}{0.5\linewidth}
      \begin{align}
          \begin{split}
                b_{600}={}&\frac{k \lambda}{\sin(\arctan(\frac{l}{h}))} \\
                &= \frac{530nm}{\sin{\arctan{\frac{34cm}{100cm}}}} \\
                &\approx \SI{1.6}{\micro\metre}
          \end{split}
      \end{align}
    \end{minipage}%
    \begin{minipage}{0.5\linewidth}
      \begin{align}
          \begin{split}
                b_{100}={}&\frac{k \lambda}{\sin(\arctan(\frac{l}{h}))} \\
                &= \frac{530nm}{\sin{\arctan{\frac{5cm}{100cm}}}} \\
                &\approx \SI{11}{\micro\metre}
          \end{split}
      \end{align}
    \end{minipage}
\end{subequations}


% 600 Linien / mm
% Leybold
% Reflexionsgitter

\documentclass[../main.tex]{subfiles}

Das Gitter beugt die Strahlung in die verschiedenen Wellenlängen. Hierfür gibt es Gitter verschiedener Linienzahlen pro mm, oder auch Reflexionsgitter, welche sowohl beugen als auch reflektieren (Abb. \ref{fig:gitter_600l}).
Folgende Formel beschreibt die position der Maxima (vielfache von $\lambda$) auf Basis der Strichbreite $b$. 
\[
    k \lambda = b \sin{\alpha}
\]

\begin{figure}[ht]
    \centering
    \begin{tikzpicture}
        %\draw[step=1,gray,dotted,thin] (0,1) grid (7,7);
        
        % Parallel Waves
        \foreach \i in {0.5,1,...,2.5}
            \draw[gray,dashed,thick] ({\i},2) -- ({\i},6);
        
        % Double Slit @ 3 & 5
        \draw[thick] (3,1.5) -- (3,2.9);
        \draw[thick] (3,3.1) -- (3,4.9);
        \draw[thick] (3,5.1) -- (3,6.5);
        
        
        \coordinate (bslit) at (3,3);
        \coordinate (tslit) at (3,5);
        \coordinate (max) at (6,6);
        \coordinate (intersect) at (4,4);
        
        % Sine waves
        \draw[gray,very thin,wavy] (tslit) -- (max);
        \draw[blue,thick] (tslit) -- (max);
        
        \draw[gray,very thin,wavy] (bslit) -- (max);
        \draw[blue,thick] (bslit) -- (max);
        
        % perpendicular
        \draw[thick,dashed] (tslit) -- (intersect);
        
        % Labeling
        \draw (3,4.7) node[anchor=north west] {$\alpha$};
        \draw (3.5,3.5) node[anchor=north west] {$\Delta x$};
        \draw[thick,<->] (2.75,2.9) -- (2.75,4.9);
        \draw (2.75,4) node[anchor=east] {$b$};
        \draw (6.2,6) node[anchor=west] {$k = 4$};
        
        % Interference pattern
        \foreach \i in {1,2,...,20}
            \shade[black,white,shading angle={mod(\i,20)*180}] (6,{\i/4+1.25}) rectangle (6.2,{\i/4+1.5});
    \end{tikzpicture}
    \caption{Grafische Darstellung der Formel}
    \label{fig:doppelspalt_beugung_berechnen}
\end{figure}

\begin{subequations}
    \noindent\begin{minipage}{0.5\linewidth}
      \begin{align}
          \begin{split}
              b_{600}={}&\frac{k \lambda}{\sin{\alpha}} \\
              &= \frac{530nm}{\sin{\arctan{\tan{\alpha}}}} \\
              &= \frac{530nm}{\sin{\arctan{\frac{34cm}{100cm}}}} \\
              &\approx \frac{530nm}{\sin{\SI{0.328}{\degree}}} \\
              &\approx \SI{1.6}{\micro\metre}
          \end{split}
      \end{align}
    \end{minipage}%
    \begin{minipage}{0.5\linewidth}
      \begin{align}
          \begin{split}
              b_{100}={}&\frac{k \lambda}{\sin{\alpha}} \\
              &= \frac{530nm}{\sin{\arctan{\tan{\alpha}}}} \\
              &= \frac{530nm}{\sin{\arctan{\frac{5cm}{100cm}}}} \\
              &\approx \frac{530nm}{\sin{\SI{0.05}{\degree}}} \\
              &\approx \SI{11}{\micro\metre}
          \end{split}
      \end{align}
    \end{minipage}
\end{subequations}


% 600 Linien / mm
% Leybold
% Reflexionsgitter

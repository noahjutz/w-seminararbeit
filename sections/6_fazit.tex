\documentclass[../main.tex]{subfiles}

% 6.1 Intention
Vorhaben dieses Projekts war die Entwicklung eines Infrarotspektrometers. Das heißt, ein Versuch, ein IR-Spektrometer zu bauen, ein Spektrum aufzunehmen, und den Verlauf zu protokollieren. Zusätzlich war ein Ziel, mithilfe eines Arduinos das Spektrum deutlicher zu machen. Die Absicht war es, den Sensor motorgesteuert mit einer konstanten Geschwindigkeit zu bewegen, und dabei das Signal aufzunehmen.

% 6.2 Herausforderungen
Die hauptsätzliche Herausforderung bei der Entwicklung eines IR-Spektrometers ist, dass die Wellen, mit denen gearbeitet wird, nicht für das menschliche Auge sichtbar sind. Wir können also nicht erkennen, was für ein Muster sich bei der Interferenz ergibt, sondern müssen uns auf eine Thermosäule berufen, welche möglicherweise aufgrund von vielen äußeren Einflussfaktoren kein klares Signal abgibt.

% 6.3 Vorgehen
Dennoch konnten jegliche Hürden durch die Anwendung physikalischer Formeln überwunden werden. So ließen sich die Winkel der Maxima bei der Interferenz eines Lötkolbens am Gitter berechnen (Gl. \ref{eq:alpha_max1_loetkolben}), um das unsichtbare Muster vorherzusagen.

% 6.4 Unerwartete Ergebnisse
Schlussendlich war bei der Auswertung des Spektrums kein erwartetes Ergebnis zu erkennen. Dazu kommt, dass die Arduinosteuerung nicht realisiert wurde.
% 6.4.1 Grund
Zu den Mitwirkende Faktoren der Nichtvollendung eines erfolgreichen IR-Spektrometers gehören u. A. Zeit- und Ressourcenmangel: Das geeignete Gitter stand mir zunächst nicht zur Verfügung, und die Arduino-Geräte (Schrittmotor, Lineare Schiene, etc.) haben für den Bau des Arduinogesteuerten Sensors gefehlt.
% 6.4.2 Lösungen
Aber, um das Beste daraus zu machen, entwurf ich die theoretischen Pläne zur Entwicklung eines solchen Geräts. Das könnte immerhin dem nächsten helfen, erfolgreich ein IR-Spektrometer zu bauen.

% 6.5 Erfolge
Das Experiment war insofern erfolgreich, dass ich höchstwarscheinlich mit etwas zusätzlicher Zeit ein Aussagekräftigen Spektrum aufnehmen könnte. Angefangen mit fehlendem grundlegendem Physik-wissen, bin ich an einem Punkt angekommen, an dem ich die Feinheiten der IR-Spektroskopie verfolgen kann.

\documentclass[../main.tex]{subfiles}

% 6.1 Intention
Vorhaben dieses Projekts war die Entwicklung eines Infrarotspektrometers. Das heißt, ein Versuch, ein IR-Spektrometer zu bauen, ein Spektrum aufzunehmen, und den Verlauf zu protokollieren. Zusätzlich war ein Ziel, mithilfe eines Arduinos das Spektrum deutlicher zu machen. Die Absicht war es, den Sensor motorgesteuert mit einer konstanten Geschwindigkeit zu bewegen, und dabei das Signal aufzunehmen.

% 6.2 Herausforderungen
Die hauptsätzliche Herausforderung bei der Entwicklung eines IR-Spektrometers ist, dass die Wellen, mit denen gearbeitet wird, nicht für das menschliche Auge sichtbar sind. Wir können also nicht erkennen, was für ein Muster sich bei der Interferenz ergibt, sondern müssen uns auf eine Thermosäule berufen, welche möglicherweise aufgrund von vielen äußeren Einflussfaktoren kein klares Signal abgibt.

% 6.3 Vorgehen
Dennoch konnten jegliche Hürden durch die Anwendung physikalischer Formeln überwunden werden. So ließen sich die Winkel der Maxima bei der Interferenz eines Lötkolbens am Gitter berechnen (Gl. \ref{eq:alpha_max1_loetkolben}), um das unsichtbare Muster vorherzusagen.

% 6.4 Miserfolge? Inwiefern? Grund? Pot. Lsg?

% 6.5 Erfolge
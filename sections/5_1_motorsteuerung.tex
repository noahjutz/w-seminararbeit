\documentclass[../main.tex]{subfiles}

Um eine möglichst genau Messung zu erheben, verwenden wir einen Arduinogesteuerten Motor, um den Sensor hin und her zu fahren. Befestigen wir den Sensor an eine lineare Schiene, welche von einem Schrittmotor gefahren wird, erhalten wir die besten Ergebnisse.

Wir gehen im Programm von einem Schrittmotor mit 4 verbundenen Anschlüssen auf Pin 8, 9, 10 und 11 aus. Darüber hinaus benötigt er 200 Schritte pro Umdrehung, in denen die Schiene um \SI{20}{\milli\metre} bewegt wird. Die Schiene ist insgesamt \SI{150}{\milli\metre} lang. Wir wählen arbiträr eine Geschwindigkeit von 60 Umdrehungen pro Minute.

\begin{figure}[ht]
\begin{lstlisting}[language=C++]
const int rpm = 60;
const int stepsPerRevolution = 200;
const int measurementRange = 150;
const int mmPerRevolution = 20;
\end{lstlisting}
\caption{Konstanten bzgl. Schrittmotors}
\end{figure}

Um den Schrittmotor einfach zu kontrollieren, verwenden wir die Arduino Stepper Library, welche mit Angabe einer Geschwindigkeit und Schrittanzahl die richtigen Signale an die Motoranschlüsse weitergibt. Die Gesamtanzahl an Schritten berechnet sich folgendermaßen:

\begin{equation}
    \text{totalSteps} = \frac{\text{measurementRange}}{\text{mmPerRevolution}} \cdot \text{stepsPerRevolution}
\end{equation}

\begin{figure}[ht]
\begin{lstlisting}[language=C++]
#include <Arduino.h>
#include <Stepper.h>
Stepper stepper(stepsPerRevolution, 8, 9, 10, 11)
void move() {
    stepper.setSpeed(rpm);

    int totalSteps = (measurementRange / mmPerRevolution) * stepsPerRevolution;
    
    stepper.step(totalSteps);
    stepper.step(-totalSteps);
}
\end{lstlisting}
\caption{Programm zur Motorsteuerung}
\end{figure}
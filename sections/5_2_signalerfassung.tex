\documentclass[../main.tex]{subfiles}

Um das Signal des Sensors aufzunehmen und über den Arduino an den angeschlossenen PC weiterzuleiten, verwenden wir die \verb|Serial| API \cite{arduino_serial_lib} der Arduino Programmiersprache. Wir setzen die Datenrate der seriellen Übertragung arbiträr auf 9600, weil es für unsere Messungen ausreichen wird. Stecken wir einen Sensor mit einer Ausgabe von 0-5V auf Pin A0 ein, können wir das Signal folgendermaßen messen: Der Output erscheint am Bildschirm des angeschlossenen PCs (Abb. \ref{fig:platformio-signal-simple}).
\begin{figure}[ht]
\begin{lstlisting}
Serial.begin(9600);
Serial.println(String(analogRead(A0)))
\end{lstlisting}
\caption{Programm zur Signalerfassung}
\label{fig:programm-signalerfassung}
\end{figure}


\documentclass[../main.tex]{subfiles}

Zunächst benötigen wir eine Infrarot-Strahlquelle, welche die Substanz durchdringt und danach mit einem Detektor gemessen wird. Als Strahlquelle bietet sich ein Lötkolben an (Abb. \ref{fig:loetkolben}), da er bei einer Temperatur von $T = 541K$ Strahlungen in Infrarotbereich abgibt. Das Strahlmaximum lässt sich mit folgender Formel berechnen:

\begin{equation}
    \lambda * T = b
\end{equation}


% λ × T = b
% Lötkolben

\documentclass[../main.tex]{subfiles}

Zunächst benötigen wir eine Infrarot-Strahlquelle, welche die Substanz durchdringt und danach mit einem Detektor gemessen wird. Als Strahlquelle bietet sich ein Lötkolben an (Abb. \ref{fig:loetkolben}), da er bei einer Temperatur von $T = 541K$ Strahlungen in Infrarotbereich abgibt. Um das zu bestätigen, berechnen wir mit folgender Formel das Strahlmaximum $\lambda_{\text{max}}$ in Abhängigkeit der Temperatur $T$ und der ??? $b$.

\begin{equation}
    \lambda_{\text{max}} \cdot T = b
\end{equation}
\begin{equation}
    \begin{split}
        \lambda_{max}={}&\frac{b}{T} \\
        &=\frac{2.8 \cdot 10^{-3}\text{mK}}{\SI{541}{\kelvin}} \\
        &\approx 0.7 \cdot 10^{-5} \\
        &\approx 7 \cdot 10^{-6} \\
        &\approx \SI{7}{\micro\metre} \\
    \end{split}
\end{equation} % TODO wrong result? should be 5micrometre?

Mit \SI{7}{\micro\metre} liegt $\lambda_{\text{max}}$ des Lötkolbens im Infrarotbereich. Somit ist er als Strahlquelle geeignet.

% λ × T = b
% Lötkolben

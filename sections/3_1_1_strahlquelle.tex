\documentclass[../main.tex]{subfiles}

Zunächst benötigen wir eine Infrarot-Strahlquelle, welche die Substanz durchdringt und danach mit einem Detektor gemessen wird. Als Strahlquelle bietet sich ein Lötkolben an (Abb. \ref{fig:loetkolben}), da er bei einer Temperatur von $T = 541K$ Strahlungen in Infrarotbereich abgibt. Das Strahlmaximum $b$ einer Welle bestimmter Wellenlänge $\lambda$ lässt sich mit folgender Formel anhand der Temperatur $T$ berechnen:
\begin{equation}
    \lambda * T = b
\end{equation}
Setzen wir die infrarote Wellenlänge eines Lötkolbens $\lambda = \SI{10}{\micro\metre}$ und seine Temperatur $T = \SI{541.15}{\kelvin}$ ein, so erhalten wir:
\begin{equation}
    \SI{10}{\micro\metre} * \SI{541.15}{\kelvin} = \SI{5411.5}{\milli\kelvin}
\end{equation}


% λ × T = b
% Lötkolben

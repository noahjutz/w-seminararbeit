\documentclass[../main.tex]{subfiles}

Führen wir das Programm aus Abb. \ref{fig:programm-signalerfassung-und-motorsteuerung} aus, erwarten wir eine Ausgabe von Werten in folgender Art:
\begin{lstlisting}
1    1542
2    1244
3    1242
4    3244
...
\end{lstlisting} % TODO real data

Die soeben erfassten Daten lassen sich mithilfe \verb|pgfplots| und \TeX\ graphisch darstellen.

\begin{figure}[ht]
    \begin{subfigure}[b]{0.5\textwidth}
\begin{lstlisting}[language=TeX]
\begin{tikzpicture}
    \begin{axis}[
        xmin=0, xmax=150,
        ymin=0, ymax=1023,
    ]

    \addplot[color=blue] table {
        1    1542
        2    1244
        % ...
    }
\end{tikzpicture}
\end{lstlisting}
    \end{subfigure}
    \begin{subfigure}[b]{0.5\textwidth}
        \begin{tikzpicture}
            \begin{axis}[
                xlabel={Messlänge [mm]},
                ylabel={Transmission [Arbiträre Einheit]},
                xmin=0, xmax=150,
                ymin=0, ymax=1023,
                width=\textwidth,
                height=\textwidth
            ]
            \end{axis}
            \begin{scope}[x={185},y={185}]
                \draw (0,0) .. controls (0.15,0.25) .. (0.3,0);
                \draw (0.3,0) .. controls (0.45,0.75) .. (0.6,0);
                \draw (0.6,0) .. controls (0.75,0.25) .. (0.9,0);
            \end{scope}
        \end{tikzpicture}
    \end{subfigure}
    \caption{Graphische Darstellung der Werte mit \TeX}
\end{figure}
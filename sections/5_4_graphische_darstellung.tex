\documentclass[../main.tex]{subfiles}

Führen wir das Programm aus Abb. \ref{fig:programm-signalerfassung-und-motorsteuerung} aus, erwarten wir eine Ausgabe von Werten in folgender Art:
\begin{lstlisting}
1    0
2    20
3    80
...
\end{lstlisting}
Die linken Werte sind die Messlängen von \SI{0}{\milli\metre} bis \SI{5}{\milli\metre}, die rechten die aufgenommene elektrische Spannung von \SI{0}{\volt} bis \SI{5}{\volt}, linear auf einen Zahlenbereich von 0-1023 übertragen.

Die soeben erfassten Daten lassen sich mithilfe \verb|pgfplots| und \TeX\ graphisch darstellen. Es müssen nur die Tab-Seperierten Werte in ein \verb|tikzpicture| eingefügt werden. Das resultat des Programms in Abb. \ref{fig:latex-visualisierung} lässt sich in Abb. \ref{fig:ergebnisse-erwartetes-signal} sehen.

\begin{figure}[ht]
\begin{lstlisting}[language=TeX]
\begin{tikzpicture}
    \begin{axis}[
        xmin=0, xmax=150,
        ymin=0, ymax=1023,
    ]
        \addplot table {
            1    0
            2    20
            % etc ...
        }
    \end{axis}
\end{tikzpicture}
\end{lstlisting}
    \caption{Graphische Darstellung der Werte mit \TeX}
    \label{fig:latex-visualisierung}
\end{figure}
